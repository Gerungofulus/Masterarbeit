\definecolor{white}{rgb}{1,1,1}
\definecolor{black}{rgb}{0,0,0}
\definecolor{darkgreen}{rgb}{0,.4,0}
\definecolor{darkred}{rgb}{0.6,0.1,0.1}
\definecolor{red}{rgb}{1,0,0}
\definecolor{green}{rgb}{0,1,0}
\definecolor{blue}{rgb}{0,0,1}
\definecolor{gray}{rgb}{0.5,0.5,0.5}
\definecolor{lightgray}{rgb}{0.85,0.85,0.85}
\definecolor{darkgray}{rgb}{0.25,0.25,0.25}
\definecolor{orange}{rgb}{1,0.8,0.1}

\lstset{
  basicstyle=\ttfamily,            % Code font, Examples: \footnotesize, \ttfamily
  breakatwhitespace=true,         % sets if automatic breaks should only happen at whitespace
  breaklines=true,                 % sets automatic line breaking
  postbreak=\space,
  captionpos=b,                    % sets the caption-position to bottom
  columns=flexible,                  % Column format
  commentstyle=\color{darkgreen},  % comment style
  escapeinside={\%*}{*)},          % if you want to add LaTeX within your code
  extendedchars=true,              % lets you use non-ASCII characters; for 8-bits encodings only, does not work with UTF-8
  frame=single,                    % adds a frame around the code
  framerule=2pt,  
  backgroundcolor=\color{lightgray}, % choose the background color; you must add \usepackage{color} or \usepackage{xcolor}
  identifierstyle=\color{black},
  keywordstyle=\color{blue},       % keyword style
  language=C++,                    % the language of the code
  numbers=left,                    % where to put the line-numbers; possible values are (none, left, right)
  numbersep=10pt,                   % how far the line-numbers are from the code
  numberstyle=\tiny\color{gray},   % the style that is used for the line-numbers
  rulecolor=\color{gray},         % if not set, the frame-color may be changed on line-breaks within not-black text (e.g. comments (green here))
  showspaces=false,                % show spaces everywhere adding particular underscores; it overrides 'showstringspaces'
  showstringspaces=false,          % underline spaces within strings only
  showtabs=false,                  % show tabs within strings adding particular underscores
  stepnumber=1,                    % the step between two line-numbers. If it's 1, each line will be numbered
  stringstyle=\color{darkred},     % string literal style
  tabsize=2,                       % sets default tabsize to 2 spaces
  caption=\lstname,                   % show the filename of files included with \lstinputlisting; also try caption instead of title  
  %
  morekeywords={__halt_compiler, abstract, and, array, as, break, callable, case, catch, class, clone, const, continue, declare, default, die, do, echo, else, elseif, else if, empty, enddeclare,
                endfor, endforeach, endif, endswitch, endwhile, eval, exit, extends, final, for, foreach, function, global, goto, if, implements, include, include_once, instanceof, insteadof,
                interface, isset, list, namespace, new, or, print, private, protected, public, require, require_once, return, static, switch, throw, trait, typedef, try, unset, use, var, while, xor, null,
                __construct
                },            % if you want to add more keywords to the set
  deletekeywords={}             % if you want to delete keywords from the given language
}





\DeclareCaptionFont{black}{\color{black}}
\DeclareCaptionFormat{listing}{#1#2#3}
\captionsetup[lstlisting]{format=listing,labelfont=black,textfont=black, singlelinecheck=false, margin=0pt, font={bf,footnotesize}}


\newcommand\TitleBackgroundPic{
\put(-4,0){
\parbox[b][\paperheight]{\paperwidth}{%
\vfill
\centering
\includegraphics[width=\paperwidth,height=\paperheight,
keepaspectratio]{title.png}%
\vfill
}}}

\newcommand\LayoutPic{
\put(-4,0){
\parbox[b][\paperheight]{\paperwidth}{%
\vfill
\centering
\includegraphics[width=\paperwidth,height=\paperheight,
keepaspectratio]{Layout.png}%
\vfill
}}}
