\section{Tested Strategies}\label{chap:testedStrategiesDesc}
In this little section the different measures of controlling the disease spread shall be discussed.
\paragraph{Strategy I - No Measures}
In order to measure the effect that the different tested strategies are having, the first strategy examined serves just for comparison.
\paragraph{Strategy II - Old BVD Regulation}
The old BVD regulation requires are newborn calves to be tested. They should be tested immediately after their birth (when they get their eartag) and if the test result is positive, they should be removed immediately (see above). The farmer can decide to wait for another test, but many federal states decided to pay a bonus towards farmers who remove cows which were tested positive within a small time frame \citep{personalCom}. The time that has to pass for a second test depends on the kind of test (see above). The old BVD regulation offered a time frame of up to $60\,\text{days}$ to retest the cow before it had to be removed. 
If a test is negative, the mother will also be marked negative, because calves of PIs are always PIs. Cows with a positive result may not be traded.
\paragraph{Strategy III - New BVD Regulation}
The new BVD regulation limits the time span to remove a cow that has been tested positive to 40 days. Additionally the farm will be put under quarantine for this time. 
\paragraph{Strategy IV - New BVD Regulation + Vaccination}
In addition to the rules of the new BVD regulation all cows need to be vaccinated. They will be vaccinated righter after 6 months when the maternal antibodies stop working and the vaccination will be applied again every year. 
\paragraph{Strategy V - New BVD regulation + Young Calf Window}
As explained above the idea of the YCW is to test the prevalence of the whole herd. In case that one of the $n$ tested cows is tested positive, the farm has to test more cows. The rules for the whole procedure are made by the federal state. This thesis focuses on Thuringia, but the rules for Bavaria (a neighbor federal state) and Austria are given for comparison.
