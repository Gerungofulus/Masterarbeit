\section{Containment Strategies}\label{chap:containmentBasics}
This thesis focuses on containment strategies in the cattle trade network of Thuringia, one of the German federal states. This chapter will elaborate on the different methods which are used to then be used for the various strategies discussed later in the chapter.

\subsection{Methods of Containment}
In real life different mechanisms will be applied together to control a specific disease. The ones interesting in the case of BVD shall be discussed in the following few paragraphs. More or less all data about these methods have been obtained through personal communication \citep{personalCom}.

\paragraph{Individual Testing}
In general a wide variety of tests to determine if a test subject carries a certain virus exist. In the case of BVD only two different strategies are applied. They are either testing on the virus itself and its parts, (virological test) or they are testing on antibodies (serological tests) \citep{haller1999diagnostik}, \citep{personalCom}. 

\subparagraph{Virological Tests} test for parts of the virus or the virus as a whole. In case of BVD they are mostly done using the cut off material from the calf's ear when they get an ear tag, but also "normal" blood samples are taken. According to \citep{personalCom} both strategies have a sensitivity of $99.8\,\%$ and a specifity of $\approx 100\,\%$. Farmers can decide to do a second test on a cow which has been tested positive once, in order to prove that it was only a transient infection. This can only be done after 21 days for another ear tag test or 42 days later after a blood test.

\subparagraph{Serological Tests} test for antibodies. Since the body has to react to the virus first, serological tests are in general a little bit slower then virological tests and they could also react to maternal antibodies, so they are not used as often.

\paragraph{Herd Testing}
Tests on BVD for individual animals only cost a few Euros per cow but since the milk price is very low, other ways of measuring the prevalence on herd level are tested, in order to lower the financial losses. 

\subparagraph{Young Calf Window (YCW)/ Jungtierfenster (JTF))}
The YCW is a very effective way of measuring the herd prevalence. Following table 1b) from \citep{flileitfaden15}, p. 15, a minimum group of $n=14$ animals needs to be tested to determine with a certainty of $95\%$ if the prevalence in the herd is below $20\%$. Table \ref{tab:certaintyPrevalenceTest} shows the number of animals $n$ of a total population $N$ that need to be tested to prove the same prevalence with a certainty of $95\%$. After testing $n$ animals with a single positive result the prevalence has to be $p>20\%$, so that it is very likely that there is a PI in the population. After this other measures such as quarantine and tests on animals to determine their health status can follow.
\begin{wraptable}{r}{0.35\textwidth}

    \begin{center}
    \begin{tabular}{|cc|}\hline
        \rowcolor{dunkelgrau} $N$          & $n $ \\\hline
                                $\leq 10  $& $8 $ \\\hline
\rowcolor{hellgrau}             $\leq 20  $& $10$ \\\hline
                                $\leq 30  $& $11$ \\\hline
\rowcolor{hellgrau}             $\leq 60  $& $12$ \\\hline
                                $\leq 180 $& $13$ \\\hline
\rowcolor{hellgrau}             $>    180 $& $14$ \\\hline 
                              
\end{tabular}

\caption[Sample Sizes For Young Calf Window]{Number of animals $n$ of a population $N$ that need to be tested to prove with a certainty of $95\%$ that the prevalence in the herd is below $20\%$ according to \protect\citep{flileitfaden15}.}
\label{tab:certaintyPrevalenceTest} 
\end{center}
\vspace{-70pt}

\end{wraptable}
\subparagraph{Bulk Milk}
For the sake of completeness another way of measuring the prevalence of a whole herd shall be mentioned here. It is possible to measure the level of BVD-specific antibodies and therefore determine if there was a new infection event in the herd recently.
\paragraph{Removal}
Cows tested positive need to be removed from the farm. Most of the time farmers will cull them when they have been tested positive, but according to Gethmann some farmers used to sell them to other countries near by which had no restriction on BVD such as the Netherlands.

\paragraph{Quarantine}
Quarantine is a well known method used to isolate and fight diseases of all kinds. People are even discussing it in the IT sector \citep{moore2003internet}. Quarantines change the structure of the network as a whole. On the other hand it has been shown by \citep{Keeling20051} that local fluctuations in the transmission are dissipated in a very short timeframe, which is why this measure of disease control is not applied on it's own.

\subsection{Tested Strategies}
In this little section the different measures of controlling the disease spread shall be discussed.
\paragraph{Strategy I - No Measures}
In order to measure the effect that the different tested strategies are having, the first strategy examined serves just for comparison.
\paragraph{Strategy II - Old BVD Regulation}
The old BVD regulation requires are newborn calves to be tested. They should be tested immediately after their birth (when they get their eartag) and if the test result is positive, they should be removed immediately (see above). The farmer can decide to wait for another test, but many federal states decided to pay a bonus towards farmers who remove cows which were tested positive within a small time frame \citep{personalCom}. The time that has to pass for a second test depends on the kind of test (see above). The old BVD regulation offered a time frame of up to $60\,\text{days}$ to retest the cow before it had to be removed. 
If a test is negative, the mother will also be marked negative, because calves of PIs are always PIs. Cows with a positive result may not be traded.
\paragraph{Strategy III - New BVD Regulation}
The new BVD regulation limits the time span to remove a cow that has been tested positive to 40 days. Additionally the farm will be put under quarantine for this time. 
\paragraph{Strategy IV - New BVD Regulation + Vaccination}
In addition to the rules of the new BVD regulation all cows need to be vaccinated. They will be vaccinated righter after 6 months when the maternal antibodies stop working and the vaccination will be applied again every year. 
\paragraph{Strategy V - New BVD regulation + Young Calf Window}
As explained above the idea of the YCW is to test the prevalence of the whole herd. In case that one of the $n$ tested cows is tested positive, the farm has to test more cows. The rules for the whole procedure are made by the federal state. This thesis focuses on Thuringia, but the rules for Bavaria (a neighbor federal state) and Austria are given for comparison.
\subparagraph{Thuringia}\label{chap:stratThuringia}
\begin{itemize}
\item $n = 15\,\text{cows}$ need to be tested (compare to table \ref{tab:certaintyPrevalenceTest}).
\item All cows need to be older then 6 months, so that they are not protected by maternal antibodies.
\item All animals need to be unvaccinated.
\item The sample group should have the same demography as the farm as a whole.
\end{itemize}
Gethmann suggested to test all animals after a positive test of one of these tested cows, since there is no single procedure that has to be taken afterwards. While all cows are tested, the farm has to be put under quarantine. If one or more animals are found to be PIs, they have to be found immediately. Calves of cows that were pregnant during this phase need to be tested at an age of 2-6 months (compare to Bavaria' procedure). Furthermore Gethmann suggested to test two subscenarios:
\begin{itemize}
\item a) testing once a year
\item b) test every $6\,\text{months}$
\end{itemize}

\subparagraph{Bavaria}
\begin{itemize}
\item $n = 5\,\text{cows}$ need to be tested. If the stable is divided into parts, at least $n = 3\,\text{cows}$ need to be tested for every section. 
\item All animals need to be unvaccinated.
\item The tested cows need to be older then 9, but younger than $24\,\text{months}$.
\end{itemize}
If all test results are negative, all cows that have been in the farm for more than $6\,\text{months}$ are marked as non-PIs, if they did arrive in the farm below the age of $24\,\text{months}$. This test has to be repeated every $5-7\,\text{months}$.
If one of the tests is positive, all 
\begin{itemize}
\item female cows between the age of $2-24\,\text{months}$ and
\item male cows between the age of $2-9\,\text{months}$
\end{itemize}
need to be examined. Calves of pregnant cows need to be tested about 2-6 months after their birth.
\subparagraph{Austria}
\begin{itemize}
\item $n = 15\,\%\text{ of all cows}$, but at least $n = 5\,\text{young cows}$ need to be tested.
\item All cows need to be older then 6 months, so that they are not protected by maternal antibodies.
\item All animals need to be unvaccinated.
\item The tested cows need to be older then 6, but younger than $24\,\text{months}$.
\item If the herd has less then 5 animals of the age, the 5 youngest cows above the age of $6\,\text{months}$  need to be tested.
\item The difference between the oldest and youngest tested cow needs to be 4 months. 
\item All tested cows need to be part of the herd for at least 3 months.
\item Cows which have been tested positive before for antibodies do not need to be tested again.
\item If there are less then $n = 5\,\text{cows}$ to be tested, all cows which fit into this scheme need to be tested.
\end{itemize}
\paragraph{Strategy VI - New BVD regulation + Vaccination + young calf window} 
This strategy tests all the techniques discussed above.
\subsection{Mathematical Implications of Containment Strategies}
All containment strategies aim to lower the probability of infection. Since the probability of transmitting the virus in the case that two individuals meet can not be reduced, the probability of infected cows meeting susceptible cows needs to be reduced. 
Testing and culling of animals, which were tested positive, reduces the number of infected (TIs and PIs), especially the number of PIs. This is basically done by increasing the death rate of both groups. 
Quarantine changes the structure of the network. It will set at least the outgoing component of the adjacency matrix of this farm to zero, thereby reducing the probability for other farms to add a TI or PI to their farm. Unfortunately the test does not detect all infecteds, so there is still a small entry rate of PIs and TIs in every farm.
The YCW is just another way of improving the certainty that a infection in the herd is detected thereby increasing the effect of the other two methods.

\section{Real-World Results of Existing Legislation}\label{chap:rlData}
For later comparison with the results of our numerical studies some results will be presented here. 

\subsection{Prevalences without any Regulation}

\paragraph{Germany}
\subsection{Regulation in Germany}\label{chap:rlDataRegulationGermany}
\paragraph{Thuringia}
The distribution of the different compartments of a single farm with PIs which have been in this farm for some time is the following:
\\
\begin{minipage}[t]{0.5\linewidth}
No PIs
    \begin{itemize}
    \item $S= 79\,\%$
\item $I= 2\,\%$
\item $P= 0\,\%$ 
\item $R= 29\,\%$
    \end{itemize}
\end{minipage}
\begin{minipage}[t]{0.5\linewidth}
PIs for a long time
    \begin{itemize}
    \item $S= 46\,\%$
\item $I= 6\,\%$
\item $P= 2\,\%$ 
\item $R= 46\,\%$.
    \end{itemize}
\end{minipage}
\\
A survey across all of Thuringia in 2010 revealed those distributions. This survey revealed that $534\,\text{PIs}$ were living in $211\,\text{farms}(\approx 2\,\%)$. of all farms.\footnote{ (Quelle) }
