\section{Networks}\label{chap:networksBasics}
To understand the spreading of disease of networks, it is nelpful to have a basic understanding of networks and their behavior. The following chapter is going to focus on the theory of networks and give some insights on other interesting topics.
\subsection{Network Basics}
The mathematical representation of a network is a graph, a schematically drawing of such a construct is shown in figure \ref{fig:simpleNetwork}. The so called graph theory is a branch of mathematics that studies graphs, which consist of nodes and edges (or links), the connections between the nodes. In our case the different nodes are farms and the links are occurring trades between those farms.
\paragraph{Adjacency Matrix}
The matrix that describes the network in the way discussed above is called adjacency matrix. Its entries $a_{ij}$ are either one or zero depending on the existence of a link from node $i$ to node $j$. Note that this contradicts the convention of \citep{BAR16} where the entries $a_{ij}$ of the adjacency matrix are 1, if there is an edge from node $i$ to node $j$.
\begin{figure}[htbp]
\centering
\noindent\includegraphics[width=0.4\linewidth,height=\textheight,
keepaspectratio]{Graph.png} 
\caption[Graph Example]{The picture shows  a simple example of a network with 6 Nodes. The arrows are showing the connections (links) between the different nodes. Some of the links are unidirectional, others bidirectional.}
\label{fig:simpleNetwork}
\end{figure}
Following this concept the adjacency matrix for the network in figure \ref{fig:simpleNetwork} is given by:
\begin{equation}
A = \left( \begin{matrix}
 0 & 0 & 0 & 1 & 1 & 0\\
 1 & 0 & 1 & 0 & 0 & 0\\
 0 & 1 & 0 & 1 & 0 & 0\\
 0 & 0 & 0 & 0 & 0 & 0\\
 1 & 0 & 0 & 0 & 0 & 0\\
 0 & 0 & 0 & 0 & 0 & 0
 \end{matrix}
 \right). \label{eq:adjMatExamp}
\end{equation}
Note that the matrix would be symmetric if all links were bidirectional. 
If certain links were stronger than others, one would attach so called weights to the vertices, which could for example represent the number of cows being traded from one farm to another. If the link represented by $a_{rs}$ had a strength (or weight) of $85$, this would be represented by $a_{rs}= 85$. 
\paragraph{Reachability}
One way of determining if node $k$ is reachable by node $l$ in $n$ steps is taking the $n$th power of $A$ and inspecting its entry $a_{lk}$. For example the potencies 
\begin{equation}
A^2 = \left( \begin{matrix}
 1 & 0 & 0 & 0 & 0 & 0\\
 0 & 1 & 0 & 2 & 1 & 0\\
 1 & 0 & 1 & 0 & 0 & 0\\
 0 & 0 & 0 & 0 & 0 & 0\\
 0 & 0 & 0 & 1 & 1 & 0\\
 0 & 0 & 0 & 0 & 0 & 0
 \end{matrix}
 \right),
 A^3 = \left( \begin{matrix}
 0 & 0 & 0 & 1 & 1 & 0\\
 2 & 0 & 1 & 0 & 0 & 0\\
 0 & 1 & 0 & 2 & 1 & 0\\
 0 & 0 & 0 & 0 & 0 & 0\\
 1 & 0 & 0 & 0 & 0 & 0\\
 0 & 0 & 0 & 0 & 0 & 0
 \end{matrix}
 \right) 
  \label{eq:adjMatExampPot}
\end{equation}
show the connections of all nodes within two and three steps. $A^2$ shows for example 
that node 4 can be reached from node two using one of two paths with length two each. It further shows that nodes 1, 2, 3 and 5 are connected to themselves using a way of length two. This is logical because each of them has at least one bidirectional edge.
Matrix $A^3$ however shows the paths with length three. For example two paths with length three connect node 2 with node 1. This is again due to the bidirectional edges.
To find out if connections with a length of $N$ or less between two nodes exist, one would calculate the reachability matrix
 %For example the 3rd power of matrix (\ref{eq:adjMatExamp}) would show that all nodes but nodes 6 are reachable by node 3 within 3 steps. If one would want to calculate all nodes that could be reached within $N$ or less steps, one would have to take the sum:
\begin{equation}
K_N = \sum_{n=0}^N A^n, \label{eq:reachability}
\end{equation}
with $A^0 = I$, the identity matrix. In the case of this network it would not make sense to calculate more than three potencies of $A$, since they do not change the reachability. The bidirectional edges just create circles, which add up, but no new paths are created. To avoid this and therefore to make sure that only unique paths are identified, the non-zero entries of the different reachability matrices are set to one. By comparing $K_N$ with $K_{N+1}$, one can see if new path are added. 

\paragraph{Disease Spread On Networks}
As shown in figure \ref{fig:disSpreadExampl}, a given disease can spread from one node to another in case there is a link between them. For example the disease can spread from node 3 to nodes 4 and 2 in the first time step. Node 1 will soon thereafter be infected by node 2, but not by node 4. If the structure of the network stays the same, node 6 will never be infected.
\begin{figure}[htbp]

\begin{minipage}{0.5\textwidth}
\centering
\noindent\includegraphics[width=0.9\linewidth,height=\textheight,
keepaspectratio]{Graph2.png} 
\end{minipage}
\begin{minipage}{0.5\textwidth}
\centering
\noindent\includegraphics[width=0.9\linewidth,height=\textheight,
keepaspectratio]{Graph3.png} 
\end{minipage}
\caption[Infection Spread On A Static Network]{This picture shows the spread of a disease along the links between the nodes. Red nodes are infected, green ones are still susceptible. The third node is infected first. Nodes 4 and 2 get infected after time step 1.}
\label{fig:disSpreadExampl}
\end{figure}

\paragraph{Disease Spread on Time-dependent Networks.}
When it comes to applications for the basics discussed above, one has to find networks in real life. Unfortunately most networks in real life are not as simple and static as the examples shown above. 
If the networks changes it's structure over time, the adjacency matrix becomes time dependent:
\begin{equation}
A = A(t).
\end{equation}
This implies that the reachability in a time dependant network also changes over time and therefore depends on the start time. The potencies as described above would need to be replaced by the product of the adjacency matrix at each time:
\begin{equation}
A^n \Rightarrow \prod_{\tau=\tau_0}^{t}A(\tau).
\end{equation}
To calculate the reachability for all nodes during the time interval $\left[t_0,t\right]$ the sum over all these products as follows:
\begin{equation}
K(t,t_0) = \sum_{\tau_0 = t_0}^{t} \prod_{\tau=\tau_0}^{t}A(\tau).
\end{equation}
Figure \ref{fig:timeDependantNetworkSpread} illustrates the time dependency of disease spreading on time dependant networks. It is a sequel to the disease spreading shown in figure \ref{fig:disSpreadExampl}. It shows the state two time steps later. The yellow nodes recovered while the disease kept on spreading.
The network only changes slightly but makes it impossible for the disease to spread from node 1 to node 5. Instead it infects node 6 which would have stayed unaffected if the network would have stayed in it's earlier state. Even if the network would go back to it's earlier configuration the disease would not spread from node 1 to node 5 because node 1 will already have recovered in the next time step and therefore be unable to infect any other node. The only possibility for the disease to reach node 5 would be an upcoming link from node 6 to node 5.
\begin{figure}[htbp]
\begin{minipage}{0.5\textwidth}
\centering
\noindent\includegraphics[width=0.9\linewidth,height=\textheight,
keepaspectratio]{Graph4.png} 
\end{minipage}
\begin{minipage}{0.5\textwidth}
\centering
\noindent\includegraphics[width=0.9\linewidth,height=\textheight,
keepaspectratio]{Graph5.png} 
\end{minipage}
\caption[Disease Spread On A Time Dependant Network]{The two graphs show the difference a time-varying network can make for the spread of a disease on it. It is illustrating the state of the network to time steps later than in the figure \ref{fig:disSpreadExampl}. The yellow nodes are recovered. In the left hand picture node 1 infects node 5. In the picture on the right hand side node 1 infects node 6 instead. Even if the network would go back to its earlier state, node 1 would be recovered (yellow) in the next time step and therefore be unable to infect node 5. Note that node 4 cannot be infected by node 1 anymore because it had been infected earlier and became recovered and therefore immune in the mean time.}
\label{fig:timeDependantNetworkSpread}
\end{figure}

\subsection{Networks of Stochastic Oscillators}
It is possible to analytically predict the behavior of a whole system of many nodes with a basic SIR-Model relatively accurately, as it was shown by \citep{Keeling20051}. \citep{ROZ11} goes even further and extends this to networks of oscillators with noise. The system described by them was a network of $n$ cities which exposed a fixed point. In contrast to the examples given above this means that the node is not either infected or susceptible as a whole, but every node is split up into different compartments. There is a probability for an individual to travel from one city to another. This leads to a statistically oscillating system. The fluctuations in the system could be described by a power spectrum derived from a Fokker-Planck equation. 
The eigenvalues of the system turn out to be quite close to the eigenvalues of a single city. The phenomenological explanation for this is that the cities behave like a single city, if the coupling and therefore the mixing is relatively strong. They behave like single cities, if the coupling is weak. 
Transferring this result on the case of BVD implies that the stable state of the whole system would be close to the stable state of a single farm. The distributions for single farms can be found in chapter \ref{chap:rlDataRegulationGermany}. However these results require a network which allows for all farms to at least get infected once. Therefore it might not be applicable to our network since some farms do not necessarily buy cows. The most remarkable part about the findings from \citep{ROZ11} is that the local stochastic oscillations in the different compartments have no effect on the system as a whole, as long as the stochastic oscillations are inhibited by white noise. This would explain, why the standard SIR model can be used to describe networks of populations as mentioned in \citep{Keeling20051}. \citep{witten2007simulations} however also investigated the behavior of standard SIR models and found that depending on the kind of network inspected, the global system behavior can also expose a bigger difference when compared to the analytical SIR model. 