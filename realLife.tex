\subsection{Relevant Basics Of Statistics}
This small section is included to refresh the reader's memory regarding some basic statistical quantities. 
\paragraph{Sensitivity}
The sensitivity or true positive rate is the amount individuals $r_\text{p}$ who are tested positive and actually infected divided by sum of $r_\text{p}$ and $f_\text{n}$, individuals which have been tested negative even though they were positive:
\begin{equation}
P(\text{positive} | \text{test positive}) = \frac{r_\text{p}}{r_\text{p} + f_\text{n}}.
\end{equation}

\paragraph{Specificity}
Just as the sensitivity gives the rate with which a test correctly identifies positive test subject, the specificity or true negative rate gives the rate with which a negative test subject $r_\text{n}$ is correctly identified as negative:
\begin{equation}
P(\text{negative} | \text{test negative}) = \frac{r_\text{n}}{r_\text{n} + f_\text{p}}.
\end{equation}