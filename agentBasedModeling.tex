\subsection{Agent-Based Modeling}\label{chap:agentBasedModelBasics}

\subsubsection{Basics}
\subsubsection{Agent-Based versus Population-Based Modeling}
The compartmental model as it is described above, builds on a few assumptions. To be able to describe a system in such a way using these differential equations, the functions of the different compartments need to be not only continuous, but also differentiable. This is a very strong assumption, which is only met if the systems size is sufficiently big. As pointed out in the Introduction of \citep{bosse2012comparative}, different academic communities disagree on which model suits different scenarios best. Agent-based and population-based models were run multiple times and compared. It could be shown that for the modeling of infectious diseases small systems expose a big variance in their results whereas bigger systems seem to be more stable. Still it could also be shown that even for bigger systems the results of a statistical model as discussed above and an agent-based model do not need to agree. For another system type to be discussed, an economical system, the results were inverse. The author concluded that the choice of the modeling approach to be taken needs to be carried out carefully considering the special traits of the studied system.