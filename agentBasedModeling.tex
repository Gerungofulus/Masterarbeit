\subsection{Agent-Based Modeling}\label{chap:agentBasedModelBasics}
Another way of describing a system is the so called agent-based approach. When a system is modelled this way, it is split up into the smallest logical units, the so called agents, which then interact with one another. The agents are basically state machines. This means they possess a certain state, which influences their behavior. For example a recovered individual meeting an infected would react differently then a susceptible individual. The latter one would react by getting infected whereas the recovered individual would stay the same. This approach is especially useful for diseases which discriminate different types of individuals. For example \citep{tang98} showed a discrimination of Alzheimer Disease between whites, Hispanics and African Americans and even non discriminating diseases such as cardio-vascular diseases can have different impact on different demographies depending on social differences \citep{DinDzietham2004449} or different popularity of individuals \citep{saravanan2013mobile}. 
But also for non discriminative diseases it can make sense to use agent based approaches, since their results don't necessarily need to agree with the results by models such as the one above.
The compartmental model as it is described above, builds on a few assumptions. To be able to describe a system in such a way using these differential equations, the functions of the different compartments need to be not only continuous, but also differentiable. This is a very strong assumption, which is only met if the systems size is sufficiently big. As pointed out in the Introduction of \citep{bosse2012comparative}, different academic communities disagree on which model suits different scenarios best. Agent-based and population-based models were run multiple times and compared. It could be shown that for the modeling of infectious diseases small systems expose a big variance in their results whereas bigger systems seem to be more stable. Still it could also be shown that even for bigger systems the results of a statistical model as discussed above and an agent-based model do not need to agree. For another system type to be discussed, an economical system, the results were inverse. The author concluded that the choice of the modeling approach to be taken needs to be carried out carefully considering the special traits of the studied system.