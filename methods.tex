\subsection{Methods of Containment}
In real life different mechanisms will be applied together to control a specific disease. The ones interesting in the case of BVD shall be discussed in the following few paragraphs. More or less all data about these methods have been obtained through personal communication \citep{personalCom}.

\paragraph{Individual Testing}
In general a wide variety of tests to determine if a test subject carries a certain virus exist. In the case of BVD only two different strategies are applied. They are either testing on the virus itself and its parts, (virological test) or they are testing on antibodies (serological tests) \citep{haller1999diagnostik}, \citep{personalCom}. 

\subparagraph{Virological Tests} test for parts of the virus or the virus as a whole. In case of BVD they are mostly done using the cut off material from the calf's ear when they get an ear tag, but also "normal" blood samples are taken. According to \citep{personalCom} both strategies have a sensitivity of $99.8\,\%$ and a specifity of $\approx 100\,\%$. Farmers can decide to do a second test on a cow which has been tested positive once, in order to prove that it was only a transient infection. This can only be done after 21 days for another ear tag test or 42 days later after a blood test.

\subparagraph{Serological Tests} test for antibodies. Since the body has to react to the virus first, serological tests are in general a little bit slower then virological tests and they could also react to maternal antibodies, so they are not used as often.

\paragraph{Herd Testing}
Tests on BVD for individual animals only cost a few Euros per cow but since the milk price is very low, other ways of measuring the prevalence on herd level are tested, in order to lower the financial losses. 

\subparagraph{Young Calf Window (YCW)/ Jungtierfenster (JTF))}
The YCW is a very effective way of measuring the herd prevalence. Following table 1b) from \citep{flileitfaden15}, p. 15, a minimum group of $n=14$ animals needs to be tested to determine with a certainty of $95\%$ if the prevalence in the herd is below $20\%$. Table \ref{tab:certaintyPrevalenceTest} shows the number of animals $n$ of a total population $N$ that need to be tested to prove the same prevalence with a certainty of $95\%$. After testing $n$ animals with a single positive result the prevalence has to be $p>20\%$, so that it is very likely that there is a PI in the population. After this other measures such as quarantine and tests on animals to determine their health status can follow. The specific set of rules which has to be followed varies between different legislators. A collection of different procedures of different (federal) states is given in chapter \ref{chap:ycwdef}.
\begin{wraptable}{r}{0.35\textwidth}

    \begin{center}
    \begin{tabular}{|cc|}\hline
        \rowcolor{dunkelgrau} $N$          & $n $ \\\hline
                                $\leq 10  $& $8 $ \\\hline
\rowcolor{hellgrau}             $\leq 20  $& $10$ \\\hline
                                $\leq 30  $& $11$ \\\hline
\rowcolor{hellgrau}             $\leq 60  $& $12$ \\\hline
                                $\leq 180 $& $13$ \\\hline
\rowcolor{hellgrau}             $>    180 $& $14$ \\\hline 
                              
\end{tabular}

\caption[Sample Sizes For Young Calf Window]{Number of animals $n$ of a population $N$ that need to be tested to prove with a certainty of $95\%$ that the prevalence in the herd is below $20\%$ according to \protect\citep{flileitfaden15}.}
\label{tab:certaintyPrevalenceTest} 
\end{center}
\vspace{-70pt}

\end{wraptable}
\subparagraph{Bulk Milk}
For the sake of completeness another way of measuring the prevalence of a whole herd shall be mentioned here. It is possible to measure the level of BVD-specific antibodies and therefore determine if there was a new infection event in the herd recently.
\paragraph{Removal}
Cows tested positive need to be removed from the farm. Most of the time farmers will cull them when they have been tested positive, but according to Gethmann some farmers used to sell them to other countries near by which had no restriction on BVD such as the Netherlands.

\paragraph{Quarantine}
Quarantine is a well known method used to isolate and fight diseases of all kinds. People are even discussing it in the IT sector \citep{moore2003internet}. Quarantines change the structure of the network as a whole. On the other hand it has been shown by \citep{Keeling20051} that local fluctuations in the transmission are dissipated in a very short timeframe, which is why this measure of disease control is not applied on it's own.
